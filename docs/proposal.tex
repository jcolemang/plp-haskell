
\documentclass{general}

%\usepackage[left=1.5in, right=1.5in, top=1.5in, bottom=1.5in]{geometry}

\settitle{Project Proposal}
\setname{James Gibson, Kieran Groble, Jenna Wohlpart}
\setclass{Programming Language Paradigms}
\setdate{\today}

\begin{document}
\createtitlepage{}

\section*{Proposal}

Our final project will be to implement the Bently-Ottmann Algorithm to detect
line segment intersections. A large part of the project will be to implement the
supporting data structures on our own using dependent types in Haskell. One
major reason that Haskell is well suited for the problem is Haskell's dependent
types. With dependent types we can enforce the validity of the red-black tree
which will be backing our algorithm. Another reason that Haskell will be well
suited is that our project is clearly a function which takes in line segments
and returns the crossing segments. We will not have to fight with its lack of
state as a result. In addition, we will see what we can learn from other parts
of the language. In particular, we hope to explore the type classes which are
missing from Elm.

Because the world is not purely functional, we will also include file I/O and a
visualization of the intersecting line segments. This will give us a more ``real
world'' view of Haskell as a general purpose language.

Lastly, we will implement a number of unit tests to verify that our data
structures and algorithm work as expected. This will allow us to see some of the
benefits of purely functional languages, as our unit tests will not need to
include any state management.

Many parts of the project will be coded in trios as we feel that they are all
critical to our understanding. We do not feel the red-black tree should be done
alone by one person as we would each like to use the dependent types. We do not
feel the algorithm should be broken apart as we would each like to understand
the final algorithm, although focus can be split by the person that wrote the
corresponding supporting data structure. The rest will be broken apart as
necessary.

\begin{table}[h]
  \centering
  \begin{tabular}{lcc}
    \textbf{Red-Black Trees} & & \\
    Lookup and Structure & 10 & All \\
    Insertion & 5 & Kieran \\
    Removal & 5 & Coleman \\
    \textbf{Priority Queue} & & \\
    Works & 10 & Jenna \\
    \textbf{Bently-Ottmann Algorithm} & & \\
    Works & 5 & All \\
    File I/O and Drawing & 5 & Coleman and Kieran \\
    \textbf{Tests} & & \\
    Completed & 5 & All \\
    \textbf{Extra} & &  \\
    Completed & 5 & ??? \\
  \end{tabular}
\end{table}


\end{document}